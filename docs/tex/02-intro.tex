\chapter*{Введение}
\addcontentsline{toc}{chapter}{Введение}

Работая с операционной системой Linux \cite{linux}, пользователю может потребоваться отслеживать её загруженность. Для обнаружения и предотвращения сбоев необходимо иметь хорошую систему мониторинга, которая будет анализировать работу операционной системы. Данный курсовой проект посвящен исследованию структур ядра, хранящим информацию о процессах в системе и памяти, и способам перехвата системных вызовов ядра с их последующим логированием.

Целью данной курсовой работы является разработка загружаемого модуля ядра, предоставляющего информацию о загруженности системы: количество системных вызовов за выбранный промежуток времени, количество выделенной памяти в текущий момент, статистика по процессам и в каких состояниях они находятся.

Для достижения поставленной цели необходимо выполнить следующие задачи:

\begin{itemize}
	\item изучить структуры и функции ядра, которые предоставляют информацию о процессах и памяти;
	\item проанализировать существующие подходы к перехвату системных вызовов и выбрать наиболее подходящий;
	\item реализовать загружаемый модуль ядра.
\end{itemize}